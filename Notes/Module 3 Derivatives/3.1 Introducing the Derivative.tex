\documentclass[12pt]{article}

\usepackage[margin=1in]{geometry}
\usepackage{parskip,pgfplots,amsmath,amsthm,amssymb,graphicx,mathtools,tikz,hyperref}
\usetikzlibrary{positioning}
\pgfplotsset{compat = newest}
\newcommand{\n}{\mathbb{N}}
\newcommand{\z}{\mathbb{Z}}
\newcommand{\q}{\mathbb{Q}}
\newcommand{\cx}{\mathbb{C}}
\newcommand{\real}{\mathbb{R}}
\newcommand{\field}{\mathbb{F}}
\newcommand{\ita}[1]{\textit{#1}}
\newcommand{\com}[2]{#1\backslash#2}
\newcommand{\oneton}{\{1,2,3,...,n\}}
\newcommand\idea[1]{\begin{gather*}#1\end{gather*}}
\newcommand\ef{\ita{f} }
\newcommand\eff{\ita{f}}
\newcommand\proofs[1]{\begin{proof}#1\end{proof}}
\newcommand\inv[1]{#1^{-1}}
\newcommand\setb[1]{\{#1\}}
\newcommand\en{\ita{n }}
\newcommand{\vbrack}[1]{\langle #1\rangle}

\newenvironment{theorem}[2][Theorem]{\begin{trivlist}
\item[\hskip \labelsep {\bfseries #1}\hskip \labelsep {\bfseries #2.}]}{\end{trivlist}}
\newenvironment{lemma}[2][Lemma]{\begin{trivlist}
\item[\hskip \labelsep {\bfseries #1}\hskip \labelsep {\bfseries #2.}]}{\end{trivlist}}
\newenvironment{exercise}[2][Exercise]{\begin{trivlist}
\item[\hskip \labelsep {\bfseries #1}\hskip \labelsep {\bfseries #2.}]}{\end{trivlist}}
\newenvironment{reflection}[2][Reflection]{\begin{trivlist}
\item[\hskip \labelsep {\bfseries #1}\hskip \labelsep {\bfseries #2.}]}{\end{trivlist}}
\newenvironment{proposition}[2][Proposition]{\begin{trivlist}
\item[\hskip \labelsep {\bfseries #1}\hskip \labelsep {\bfseries #2.}]}{\end{trivlist}}
\newenvironment{corollary}[2][Corollary]{\begin{trivlist}
\item[\hskip \labelsep {\bfseries #1}\hskip \labelsep {\bfseries #2.}]}{\end{trivlist}}
 \hypersetup{
 colorlinks,
 linkcolor=blue
 }
\begin{document}


\title{MA 211: Module 3 Notes}
\author{Patrick Rall}

\maketitle

\newpage
\section{Introducing the Derivative}
The problem of finding the slope of the tangent to a curve is important for several reasons.
\begin{itemize}
    \item We identify the slope of the tangent with the \emph{instantaneous rate of change} of a function.
\end{itemize}
\begin{align}
    \begin{tikzpicture}
        \begin{axis}[
                xtick = \empty,    ytick = \empty,
                xlabel = {$x$},
                x label style = {at={(1,0)},anchor=west},
                ylabel = {$y$},
                y label style = {at={(0,1)},rotate=-90,anchor=south},
                axis lines=left,
                enlargelimits=0.2,
            ]
            \addplot[color=red,smooth,thick,-] {(x)^2};
            \addplot[mark=none, blue] coordinates {(-6,20) (0,-4)};
        \end{axis}
    \end{tikzpicture}
\end{align}
\begin{itemize}
    \item The slopes of the tangent lines as they change along a curve are teh values of a new function called the \emph{derivative}.
    \item If a curve represents the trajectory of a moving object, the tangent at a point on a curve indicates the direction of motion at that point.
\end{itemize}

If $s(t)$ is the position of the object at time $t$, then the average velocity of the object over the time interval $[a,b]$ is
\begin{equation*}
v_{av}=\frac{s(t)-s(a)}{t-a}.
\end{equation*}

The instantaneous velocity at time $t=a$ is the limit of the average velocity as $t \to a$:
\begin{equation*}
    v_{inst}=\lim_{t \to a} \frac{s(t)- s(a)}{t-a}.
\end{equation*}

\subsection{Tangent Lines and Rates of Change}
Consider the curve $y=f(x)$ and a secant line intersecting the curve at points $P(a,f(a))$ and $Q(x,f(x))$. The difference $f(x) - f(a)$ is the change in the value of $f$ on the interval $[a,x]$, while $x-a$ is the change in $x$.


\end{document}